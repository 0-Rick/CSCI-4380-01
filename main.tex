\documentclass[11pt]{article}
\usepackage{enumerate}
\usepackage{url}
\usepackage{listings}
\usepackage{upquote,textcomp}

\setlength{\parindent}{0cm}
\setlength{\parskip}{0.3cm plus4mm minus3mm}

\oddsidemargin = 0.2in
\textwidth = 6.5 in
\textheight = 9.8 in
\headsep = -1in

\lstset{frame=tb,
  language=,
  aboveskip=3mm,
  belowskip=3mm,
  showstringspaces=false,
  columns=flexible,
  keepspaces=true,
  basicstyle={\small\ttfamily},
  numbers=none,
  numberstyle=\tiny\color{black},
  keywordstyle=\color{black},
  commentstyle=\color{black},
  stringstyle=\color{black},
  breaklines=true,
  breakatwhitespace=true,
  tabsize=3
}

% ========== Title ==========
\title{Database Systems, CSCI 4380-01 \\
\textbf{Homework \# 1 Extra Credit} \\
Due Tuesday, February 3, 2026 at 7:59:59 PM EDT}
\date{}
\begin{document}
\maketitle
\vspace*{-0.7in}

% ========== Question ==========
\noindent\rule{\linewidth}{2pt}
{\bf Question 3 [30 points].}
For the following relations, find and list all the keys.
\begin{enumerate}
    \item
      $R1(A,B,C,D,E,F,G)$, ${\cal F} = \{ BDG\rightarrow FG, AF\rightarrow DE,  EF\rightarrow A\}$
\end{enumerate}
\vspace{-2em}
\noindent\rule{\linewidth}{2pt}
% ========== Solution ==========

\noindent\textbf{Solution}

\vspace{-2.2em}
\noindent
\begin{itemize}
    \item
    To start, ensuring that the fd (functional dependency) is in it's minimal basis makes it significantly easier to find the candidate keys. The four rules for transforming a fd into a minimal basis are as follows:
    \vspace{-.4em}\begin{enumerate} [a)]
        \item Convert all fd's RHS (right-hand side) to a single attribute).
        \item Remove trivial functional dependencies ($A\rightarrow A$).
        \item No redundant dependencies;
        \\ No extraneous attributes on the LHS (left-hand side).     
    \end{enumerate}
    \item [(a)] Single attribute RHS:
        \vspace{.2em}
        \\\hspace*{4.8em}\underline{Previous}\hspace{5.6em}\underline{After}
        
        \vspace{-1em}
        \begin{itemize}
            \item[$\rightharpoonup$] 
            $\cal F$ $=\{BDG\rightarrow FG$ \hspace*{1em}$\Rightarrow$\hspace*{1em} $BDG\rightarrow F, BDG\rightarrow G$
            \\\hspace*{2.6em}$AF\rightarrow DE$
            \hspace*{1.8em}$\Rightarrow$\hspace*{1em} $AF\rightarrow D, AF\rightarrow E$
            \\\hspace*{2.6em}$EF\rightarrow A$
            \hspace*{2.6em}$\Rightarrow$\hspace*{1em} $EF\rightarrow A$ \hspace*{7em}\textit{(unchanged)}
            
            \vspace{.2em}
            \item[$\rightharpoonup$] 
            New Functional Dependency:
            \\$\Rightarrow\cal F$ $=\{BDG\rightarrow F,BDG\rightarrow G, AF\rightarrow D, AF\rightarrow E, EF\rightarrow A\}$
        \end{itemize}
    \item[(b)] Removing trivial fds:
        \begin{itemize}
            \item[$\rightharpoonup$]
            $\cal F$ $=\{BDG\rightarrow F,BDG\rightarrow G, AF\rightarrow D, AF\rightarrow E, EF\rightarrow A\}$
            \vspace{-.4em}
            \\\hspace*{7.6em} $+++++$
            \\\hspace*{7.2em} (\textit{$G$ is trivial})
            
            \vspace{.2em}
            \item[$\rightharpoonup$] 
            New Functional Dependency:
            \\$\Rightarrow\cal F$ $=\{BDG\rightarrow F, AF\rightarrow D, AF\rightarrow E, EF\rightarrow A\}$
            
        \end{itemize}
    \item[(c)] Removing redundant and extraneous attributes:
        \begin{itemize}
            \item[$\rightharpoonup$]
            $\cal F$ $=\{BDG\rightarrow F, AF\rightarrow D, AF\rightarrow E,  EF\rightarrow A\}$
            
            \item[$\rightharpoonup$]
            There is a shortcut we can take here. Each RHS attribute is produced by exactly one fd, so none can be redundant. (In other words, removing one of the LHS attributes in a fd or removing an entire fd will make the RHS attribute inaccessible.)
        \end{itemize}

    \newpage
    \item 
    Finding the keys using a minimal basis(s):
        \vspace{-.4em}
            \begin{enumerate} [a)]
                \item Find all unreachable attributes on the RHS side (cannot be derived).

                \item Discarding unreachable attributes, start with the fds with the least number of attributes on the LHS to see if all attributes are reachable. 
                \\ Once all attributes are reachable, check all other LHS fds with the same number of attributes can do the same (checking for other candidate keys).
            \end{enumerate}
            
    \item[(a)]
    Inaccessible Attributes: B,C,G (\textit{must appear in all candidate keys})
    
    \item[(b)] 
    Minimal Basis: $\cal F$ $=\{BDG\rightarrow F, AF\rightarrow D, AF\rightarrow E,  EF\rightarrow A\}$
    \\ Start by trying all fds with the least number of LHS attributes -- one attribute in this case.
    \begin{itemize}
        \item[$\rightharpoonup$] 
            Add one attribute. For the one case below, we only need to add $D$.
            \vspace{.2em}
            \\ $BDG\rightarrow F$
            \\\hspace*{2em} $BCG,D+=\{B,C,D,G,F\}$  (\textit{A and E unreachable})
            
        \vspace{.4em}
        \item[$\rightharpoonup$] 
            Add two attributes. There are no other LHS fds that require only  one attribute. 
            \vspace{.2em}
            \\ $AF\rightarrow D, AF\rightarrow E$
            \\\hspace*{2em} $BCG,AF+=\{A,B,C,D,E,F\}$  \textbf{(candidate key is found!)}
            \vspace{.2em}
            \\ $EF\rightarrow A$
            \\\hspace*{2em} $BCG,EF+=\{A,B,C,E,F,G\}$  (\textit{D unreachable})
    \end{itemize}
    
    \item \textbf{Answer: Our key is $\{A,B,C,E,F,G\}$.}
    
    
\end{itemize}

\end{document}