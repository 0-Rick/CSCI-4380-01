\documentclass[11pt]{article}
\usepackage{enumerate}
\usepackage{url}
\usepackage{listings}
\usepackage{upquote,textcomp}

\setlength{\parindent}{0cm}
\setlength{\parskip}{0.3cm plus4mm minus3mm}

\oddsidemargin = 0.2in
\textwidth = 6.5 in
\textheight = 9.8 in
\headsep = -1in

\lstset{frame=tb,
  language=,
  aboveskip=3mm,
  belowskip=3mm,
  showstringspaces=false,
  columns=flexible,
  keepspaces=true,
  basicstyle={\small\ttfamily},
  numbers=none,
  numberstyle=\tiny\color{black},
  keywordstyle=\color{black},
  commentstyle=\color{black},
  stringstyle=\color{black},
  breaklines=true,
  breakatwhitespace=true,
  tabsize=3
}

% ========== Title ==========
\title{Database Systems, CSCI 4380-01 \\
\textbf{Homework \# 1 Extra Credit} \\
Due Tuesday, February 3, 2026 at 7:59:59 PM EDT}
\date{}
\begin{document}
\maketitle
\vspace*{-0.7in}

% ========== Question ==========
\noindent\rule{\linewidth}{2pt}
{\bf Question 3 [30 points].}
For the following relations, find and list all the keys.
\begin{enumerate}
    \item
      $R1(A,B,C,D,E,F,G)$, ${\cal F} = \{ BDG\rightarrow FG, AF\rightarrow DE,  EF\rightarrow A\}$
\end{enumerate}
\vspace{-2em}
\noindent\rule{\linewidth}{2pt}
% ========== Solution ==========

\noindent\textbf{Solution}

\vspace{-2.2em}
\noindent
\begin{itemize}
    \item
    To start, ensuring that the fd (functional dependency) is in it's minimal basis makes it significantly easier to find the candidate keys. The four rules for transforming a fd into a minimal basis are as follows:
    \vspace{-.4em}\begin{enumerate} [a)]
        \item Convert all fd's RHS (right-hand side) to a single attribute).
        \item Remove trivial functional dependencies ($A\rightarrow A$).
        \item No redundant dependencies;
        \\ No extraneous attributes on the LHS (left-hand side).     
    \end{enumerate}
    \item [(a)] Single attribute RHS:
        \vspace{.2em}
        \\\hspace*{4.8em}\underline{Previous}\hspace{5.6em}\underline{After}
        
        \vspace{-1em}
        \begin{itemize}
            \item[$\rightharpoonup$] 
            $\cal F$ $=\{BDG\rightarrow FG$ \hspace*{1em}$\Rightarrow$\hspace*{1em} $BDG\rightarrow F, BDG\rightarrow G$
            \\\hspace*{2.6em}$AF\rightarrow DE$
            \hspace*{1.8em}$\Rightarrow$\hspace*{1em} $AF\rightarrow D, AF\rightarrow E$
            \\\hspace*{2.6em}$EF\rightarrow A$
            \hspace*{2.6em}$\Rightarrow$\hspace*{1em} $EF\rightarrow A$ \hspace*{7em}\textit{(unchanged)}
            
            \vspace{.2em}
            \item[$\rightharpoonup$] 
            New Functional Dependency:
            \\$\Rightarrow\cal F$ $=\{BDG\rightarrow F,BDG\rightarrow G, AF\rightarrow D, AF\rightarrow E, EF\rightarrow A\}$
        \end{itemize}
    \item[(b)] Removing trivial fds:
        \begin{itemize}
            \item[$\rightharpoonup$]
            $\cal F$ $=\{BDG\rightarrow F,BDG\rightarrow G, AF\rightarrow D, AF\rightarrow E, EF\rightarrow A\}$
            \vspace{-.4em}
            \\\hspace*{7.6em} $+++++$
            \\\hspace*{7.2em} (\textit{$G$ is trivial})
            
            \vspace{.2em}
            \item[$\rightharpoonup$] 
            New Functional Dependency:
            \\$\Rightarrow\cal F$ $=\{BDG\rightarrow F, AF\rightarrow D, AF\rightarrow E, EF\rightarrow A\}$
            
        \end{itemize}
    \item[(c)] Removing redundant and extraneous attributes:
        \begin{itemize}
            \item[$\rightharpoonup$]
            $\cal F$ $=\{BDG\rightarrow F, AF\rightarrow D, AF\rightarrow E,  EF\rightarrow A\}$
            
            \item[$\rightharpoonup$]
            There is a shortcut we can take here. Each RHS attribute is produced by exactly one fd, so none can be redundant. (In other words, removing one of the LHS attributes in a fd or removing an entire fd will make the RHS attribute inaccessible.)
        \end{itemize}

    \newpage
    \item 
    Finding the keys using a minimal basis(s):
        \vspace{-.4em}
        \begin{enumerate} [a)]
            \item Find all unreachable attributes on the RHS side (cannot be derived).

            \item Build a key starting from unreachable attributes and adding RHS attributes from $\cal F$. Check every key found to ensure it is not a superkey.
        \end{enumerate}
        
    \item
    $\cal F$ $=\{BDG\rightarrow F, AF\rightarrow D, AF\rightarrow E,  EF\rightarrow A\}$
    \item[(a)]
    Inaccessible Attributes: $B,C,G$ (\textit{must appear in all candidate keys})
    \vspace{-.8em}
    \begin{itemize}
        \item[$\rightharpoonup$]
        $BCG+=\{B,C,G\}$ (\textit{not a key})
    \end{itemize}
    \item[(b)] 
    Example: Building the key with $B,C,G$: 
    \vspace{-.8em}
    \begin{itemize}
        \item[$\rightharpoonup$] 
            ($BCG$) Let's add attribute $D$, based on the functional dependency $BDG\rightarrow F$:
            \\\hspace*{2em} $BCGD+=\{B,C,D,G\}$ (\textit{missing A, E, F})
            \vspace{.2em}
            \\ ($BCGD$) Next, try adding attribute $E$, based on the functional dependency $EF\rightarrow A$:
            \\\hspace*{2em} $BCGDE+=\{A,B,C,D,E,F,G\}$ (\textit{works!})
            \vspace{.2em}
            \\\hspace*{2em} Check if $BCGDE$ is a superkey:
            \\\hspace*{4em} $CGDE+=\{A,B,C,D,E,G\}$ (\textit{missing F})
            \\\hspace*{4em} $BGDE+=\{A,B,D,E,F,G\}$ (\textit{missing C})
            \\\hspace*{4em} $BCDE+=\{A,B,C,D,E,G\}$ (\textit{missing F})
            \\\hspace*{4em} $BCGE+=\{A,B,C,D,E,G\}$ (\textit{missing F})
            \\\hspace*{4em} $BCGD+=\{B,C,D,E,F,G\}$ (\textit{missing A})
            \vspace{.2em}
            \\\hspace*{2em} Since this cannot be reduced, we conclude that $\{B,C,D,G,E\}$ is a \textbf{minimal key}.
            \vspace{-1em}
            \\ ($BCGD$) Let's try adding $A$ based on $AF\rightarrow D, AF\rightarrow E$:
            \\\hspace*{2em} $BCGAD+=\{A,B,C,D,E,F,G\}$ (\textit{works!})
            \\\hspace*{2em} Check if $BCGAD$ is a superkey:
            \\\hspace*{4em} $CGAD+=\{A,B,C,D,E,G\}$ (\textit{missing F})
            \\\hspace*{4em} $BGAD+=\{A,B,D,E,F,G\}$ (\textit{missing C})
            \\\hspace*{4em} $BCAD+=\{A,B,C,D,E,G\}$ (\textit{missing F})
            \\\hspace*{4em} $BCGD+=\{B,C,D,E,F,G\}$ (\textit{missing A})
            \\\hspace*{4em} $BCGA+=\{A,B,C,D,E,G\}$ (\textit{missing F})
            \vspace{.2em}
            \\\hspace*{2em} $\{A,B,C,D,G\}$ is a \textbf{minimal key} as it cannot be reduced.
            \vspace{.4em}
            \\ ($BCGD$) There are no other LHS attributes in the fds that we can try adding (the other attributes already exist in $BCGD+$). This means we move back to checking $BCG$.
        \item[$\rightharpoonup$] 
            ($BCG$) Add attributes $AF$ based on the fd $AF\rightarrow D, AF\rightarrow E$:
            \vspace{0em}
            \\\hspace*{2em} However, we already found a minimal key with $AF$. We can skip this fd.
        \item[$\rightharpoonup$] 
            ($BCG$) Add attributes $EF$ based on the last fd $EF\rightarrow A$:
            \\\hspace*{2em} $BCGEF+=\{A,B,C,D,G,F,G\}=R1$  (\textit{works!})
            \vspace{.2em}
            \\Check if $BCGEF$ is a superkey:
            \\\hspace*{2em} $CGEF+=\{A,C,D,E,F\}$ (\textit{missing B})
            \\\hspace*{2em} $BGEF+=\{A,B,D,E,F\}$ (\textit{missing C})
            \\\hspace*{2em} $BCEF+=\{B,C,D,E,F\}$ (\textit{missing A})
            \\\hspace*{2em} $BCGF+=\{B,C,D,E,F\}$ (\textit{missing A})
            \\\hspace*{2em} $BCGE+=\{A,B,C,D,E\}$ (\textit{missing F})
            \vspace{.2em}
            \\ We conclude that $\{B,C,E,F,G\}$ is a \textbf{minimal key}.
    \end{itemize}
    
    \item \textbf{Answer: Our minimal keys are $\{B,C,D,G,E\},\{A,B,C,D,G\},\{B,C,E,F,G\}$.}
    
    
\end{itemize}

\end{document}